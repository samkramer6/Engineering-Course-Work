
% This LaTeX was auto-generated from MATLAB code.
% To make changes, update the MATLAB code and republish this document.

\documentclass{article}
\usepackage{graphicx}
\usepackage{color}

\sloppy
\definecolor{lightgray}{gray}{0.5}
\setlength{\parindent}{0pt}

\begin{document}

    
    

\section*{ME 4005 Lab 6 Prelab}

\begin{par}
Sam Kramer                                                    16 Apr 2022 This is the work and the plot data for the two different calibration plots for Lab 6 Prelab Will return:               1. 1 graph with the 2 calibration curves for both the               pressure gauge equation and the voltage divider equation as               well as selected points at Pmeasured = -2,-1,0,1,2 inches               H20
\end{par} \vspace{1em}
\begin{par}
The experimental setup looks like:                                                \_\_\_\_\_\_\_\_\_\_\_\_\_\_\_\_\_\_                     \_\_\_\_\_\_\_\_\_               \ensuremath{|}       MSP432       \ensuremath{|}        \_\_\_\_\_\_\_\_\_\_\ensuremath{|}           \texttt{\_\_\_\_\_\_\_\_\_\_\_\_}\_\_o SV Termial      \ensuremath{|}       \ensuremath{|}            \ensuremath{|}  Pressure \ensuremath{|}       \_\_\_\_\_\ensuremath{|}\_\_\_[] Channel 1     \ensuremath{|}       \ensuremath{|}            \ensuremath{|}   Gauge   \ensuremath{|}      \ensuremath{|} \ensuremath{|}     \ensuremath{|}   [] Channel 2     \ensuremath{|}       \ensuremath{|}          \ensuremath{|}\_\_\_\_\_\_\_\_\_\_\_\ensuremath{|}      \ensuremath{|} \ensuremath{|}     \ensuremath{|}   [] Channel 3     \ensuremath{|}       \ensuremath{|}         \ensuremath{|}                     \ensuremath{|} \ensuremath{|}     \ensuremath{|}   [] Channel 4     \ensuremath{|}       \ensuremath{|}        Vin                    \ensuremath{|} \ensuremath{|}     \ensuremath{|} o ground           \ensuremath{|}       \ensuremath{|}         \texttt{\_\_\_\_\_              \ensuremath{|} \ensuremath{|}     \ensuremath{|}\_}\_\_\_\_\_\_\_\_\_\_\_\_\_\_\_\_\_\_\ensuremath{|}       \ensuremath{|}                 \ensuremath{|}             \ensuremath{|} \ensuremath{|}       \ensuremath{|}       \ensuremath{|}                 R1            \ensuremath{|} \ensuremath{|}       \ensuremath{|}       \ensuremath{|}                 \texttt{\_\_\_Vout\_\_\_\_} \ensuremath{|}       \ensuremath{|}       \ensuremath{|}                 R2              \ensuremath{|}       \ensuremath{|}       \ensuremath{|}                 \texttt{\_\_\_\_\_\_\_\_\_\_\_\_\_}       \ensuremath{|}       \ensuremath{|}                                         \ensuremath{|}       \texttt{\_\_\_\_\_\_\_\_\_\_\_\_\_\_\_\_\_\_\_\_\_\_\_\_\_\_\_\_\_\_\_\_\_\_\_\_\_\_\_}
\end{par} \vspace{1em}
\begin{verbatim}
%%%%%%%%%%%%%%%%%%%%%%%%%%%%%%%%% Work %%%%%%%%%%%%%%%%%%%%%%%%%%%%%%%%%%%%

% -- Set up
clear; clc; format compact; close all;

% -- Parameters
Vmax = 4.5;         % Max voltage output from pressure gauge(Volts)
Vmin = 0.5;         % Min voltage input from pressure gauge(Volts)
Pmax = 5;           % Max pressure reading (in H2O)
Pmin = -5;          % Min pressure reading (in H2O)
R1 = 1800;          % Voltage divider resistance 1 (Ohms)
R2 = 1200;          % Voltage divider resistance 2 (Ohms)
P = -5:1:5;         % Pressure Vector (in H2O)

% -- Pressure gauge equation Va
    % This is the voltage generated from the pressure gauge reading the
    % pressure differential in the pitot tube
    % Follows equation:
    %   Vout = ((Vmax - Vmin)/(Pmax - Pmin))*(P - Pmin) + Vmin
Va = ((Vmax - Vmin)/(Pmax - Pmin))*(P - Pmin) + Vmin;

% -- Voltage divider equation Vb
    % This will be the voltage output that will be input into the MSP432
    % and then compiled in the data files.
    % Follows equation:
    %   Vout = (Vin * R2) / (R1 + R2)
Vb = (Va*R2)/(R1 + R2);

% -- Voltage calibration plots
plot(P,Va,'Linewidth',1.5)
    hold on
plot(P,Vb,'Linewidth',1.5)
    hold on
    grid on
    xlim([-5 5])
    ylim([0 5])
    xlabel('Pressure (inches H_2O)')
    ylabel('Voltage Output (Volts)')
    title('Voltage Output of Two Devices vs. Measured Pressure')

% -- Voltge at specified positions
V = Vb(1,4:8);
P2 = -2:2;
plot(P2,V,'ko','Linewidth',1.5)
    hold on
    legend('Pressure Gauge Voltage output (V_a)','MSP Voltage input (V_b)','Pressure at selected points')
poly = polyfit(P,Vb,1);

    % In inches H2O
slope = poly(1);
fprintf('The value of the slope is %3.3f (V/In-H2O) \n',slope)

    % In Pascals
slope = slope / 249;
fprintf('The value of the slope is %3.5f V/Pa \n',slope)
\end{verbatim}



\end{document}

